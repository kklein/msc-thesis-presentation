%%%%%%%%%%%%%%%%%%%%%%%%%%%%%%%%%%%%%%%%%
% Beamer Presentation
% LaTeX Template
% Version 1.0 (10/11/12)
%
% This template has been downloaded from:
% http://www.LaTeXTemplates.com
%
% License:
% CC BY-NC-SA 3.0 (http://creativecommons.org/licenses/by-nc-sa/3.0/)
%
%%%%%%%%%%%%%%%%%%%%%%%%%%%%%%%%%%%%%%%%%

%----------------------------------------------------------------------------------------
%	PACKAGES AND THEMES
%----------------------------------------------------------------------------------------

\documentclass{beamer}

\mode<presentation> {

% The Beamer class comes with a number of default slide themes
% which change the colors and layouts of slides. Below this is a list
% of all the themes, uncomment each in turn to see what they look like.

%\usetheme{default}
%\usetheme{AnnArbor}
%\usetheme{Antibes}
%\usetheme{Bergen}
%\usetheme{Berkeley}
%\usetheme{Berlin}
%\usetheme{Boadilla}
%\usetheme{CambridgeUS}
%\usetheme{Copenhagen}
%\usetheme{Darmstadt}
%\usetheme{Dresden}
%\usetheme{Frankfurt}
%\usetheme{Goettingen}
%\usetheme{Hannover}
%\usetheme{Ilmenau}
%\usetheme{JuanLesPins}
%\usetheme{Luebeck}
\usetheme{Madrid}
%\usetheme{Malmoe}
%\usetheme{Marburg}
%\usetheme{Montpellier}
%\usetheme{PaloAlto}
%\usetheme{Pittsburgh}
%\usetheme{Rochester}
%\usetheme{Singapore}
%\usetheme{Szeged}
%\usetheme{Warsaw}

% As well as themes, the Beamer class has a number of color themes
% for any slide theme. Uncomment each of these in turn to see how it
% changes the colors of your current slide theme.

%\usecolortheme{albatross}
%\usecolortheme{beaver}
%\usecolortheme{beetle}
%\usecolortheme{crane}
%\usecolortheme{dolphin}
%\usecolortheme{dove}
%\usecolortheme{fly}
%\usecolortheme{lily}
%\usecolortheme{orchid}
%\usecolortheme{rose}
%\usecolortheme{seagull}
%\usecolortheme{seahorse}
%\usecolortheme{whale}
%\usecolortheme{wolverine}

%\setbeamertemplate{footline} % To remove the footer line in all slides uncomment this line
\setbeamertemplate{footline}[page number] % To replace the footer line in all slides with a simple slide count uncomment this line

\setbeamertemplate{navigation symbols}{} % To remove the navigation symbols from the bottom of all slides uncomment this line
}

\usepackage{algpseudocode}
\usepackage{algorithm}

\usepackage{graphicx} % Allows including images
\usepackage{booktabs} % Allows the use of \toprule, \midrule and \bottomrule in tables
%\usepackage {tikz}
\usepackage{tkz-graph}
\GraphInit[vstyle = Shade]
\tikzset{
  LabelStyle/.style = { rectangle, rounded corners, draw,
                        minimum width = 2em, fill = yellow!50,
                        text = red, font = \bfseries },
  VertexStyle/.append style = { inner sep=5pt,
                                font = \normalsize\bfseries},
  EdgeStyle/.append style = {->, bend left} }
\usetikzlibrary {positioning}
%\usepackage {xcolor}
\definecolor {processblue}{cmyk}{0.96,0,0,0}
%----------------------------------------------------------------------------------------
%	TITLE PAGE
%----------------------------------------------------------------------------------------

\title[Short title]{Check-in} % The short title appears at the bottom of every slide, the full title is only on the title page

\author{Nathanael Roy} % Your name
\institute[UC Riverside] % Your institution as it will appear on the bottom of every slide, may be shorthand to save space
{
University of California \\ % Your institution for the title page
\medskip
}
\date{\today} % Date, can be changed to a custom date

\begin{document}

%\begin{frame}
%\titlepage % Print the title page as the first slide
%\end{frame}

%\begin{frame}
%\frametitle{Overview} % Table of contents slide, comment this block out to remove it
%\tableofcontents % Throughout your presentation, if you choose to use \section{} and \subsection{} commands, these will automatically be printed on this slide as an overview of your presentation
%\end{frame}

%----------------------------------------------------------------------------------------
%	PRESENTATION SLIDES
%----------------------------------------------------------------------------------------

%------------------------------------------------

\begin{frame}{Previous work: Context}
    \begin{itemize}
        \item 'Best Arm Identification' / 'Pure Exploration Bandits'\\
            \emph{Simple Bayesian Algorithms for Best-Arm Identification}, Daniel Russo
        \item $\theta = [.2, .3, \dots, .5, .1]$ (Bernoulli)
        \item Best arm: ${arg max}_i\ \theta_i$
        \item Possible objectives:
            \begin{itemize}
                \item Fixed \#samples: high confidence
                \item Fixed confidence: few \#samples
            \end{itemize}
        \item Conventional approaches:
            \begin{itemize}
                \item Uniform sampling: Equal effort/measurements, not equal evidence.
                \item Thompson sampling: Focus on estimated best, little evidence about alternatives.
            \end{itemize}
        \item Idea: Gather equal evidence by focusing on the 'threshold'.
    \end{itemize}
\end{frame}

\begin{frame}{Previous work: Algorithm: Top-two Thompson sampling}
Given hyperparameter $\beta$, prior $\Pi_n$ at step $n$:\\
\begin{algorithmic}
\State $B \sim Bernoulli(\beta)$
\State $\hat{\theta} \sim \Pi_n$
\State $I = argmax_i\ \hat{\theta}$
\If {$B = 0$}
    \Repeat
        \State $\hat{\theta} \sim \Pi_n$
        \State $J = argmax_i\ \hat{\theta}$
    \Until{$J \neq I$}
    \State $I = J$
\EndIf
\end{algorithmic}
Play $I$, observe reward and update priors.
\end{frame}

\begin{frame}{Previous work: Theory}
    \begin{itemize}
        \item Analysis of the convergence's exponent, i.e. $\lim_{n\to\infty} \frac{1}{n} \log \Pi_n(\Theta_{I*}^C) = \Gamma$
        \item $\psi_{n, I*} = \Pr[I_n = I^*] $
        \item Under all adaptive allocation rules that have $\lim_{n\to\infty} \psi_{n, I^*} = \beta$, TTTS with parameter $\beta$ has best exponent $\Gamma_\beta^*$. \\(Constrained optimality without tuning)
        \item TTTS can be tuned to $\beta^* = argmax_\beta \Gamma_\beta^*$ and hence have the best overall possible exponent.\\ (Optimality with tuning)
        \item $\Gamma^* \leq 2 \Gamma_{\frac{1}{2}}^*$, $\frac{\Gamma^*}{\Gamma_\beta^*} \leq max\{\frac{\beta^*}{\beta}, \frac{1-\beta^*}{1 - \beta}\}$ \\(Robustness)
    \end{itemize}
\end{frame}


\begin{frame}{Project motivation}
    \begin{itemize}
        \item Best arm identification in practice often comes in different, more general shapes.
            \begin{itemize}
                \item \textbf{Top-m arms} (set)
                \item Batch updates
                \item Contextual arms
                \item Correlated arms
            \end{itemize}
    \end{itemize}
\end{frame}

\begin{frame}{Project goal}
    \begin{itemize}
        \item Define top-m arms algorithm.
        \item Determine and prove best possible convergence rate for top-m, $\Gamma^*$.
        \item Determine and prove best possible convergence rate for 'constrained' class of algorithms, $\Gamma_\gamma^*$.
        \item Prove that TTTS belongs to one such class.
        \item Prove that TTTS is optimal in said class.
        \item Prove 'robustness', i.e. relationship between constrained and non-constrained optimum.
    \end{itemize}
\end{frame}

\begin{frame}{Project: So far}
    \begin{itemize}
        \item Algorithm definition (not definite).
        \item Attempt to understand both Russo's proofs and the interaction between them.
            \begin{itemize}
                \item Bottom-up
                \item Top-down
                    \begin{itemize}
                        \item Breadth-first
                        \item Depth-first
                    \end{itemize}
            \end{itemize}
        \item Simulation.
    \end{itemize}
\end{frame}

\begin{frame}{Current algorithm}
    Given a prior $\Pi_n$ at step $n$:
    \begin{algorithmic}
    \State $\hat{\theta} \sim \Pi_n$
    \State $I_m = $ top-m$(\theta)$
    \Repeat
        \State $\hat{\theta} \sim \Pi_n$
        \State $J_m = $ top-m$(\hat{\theta})$
    \Until{$I_m \neq J_m$}
    \State $I \sim \mathcal{U}(I_m \oplus J_m)$
    \end{algorithmic}
    Play $I$, observe reward and update priors.
\end{frame}



\begin{frame}{Simulation: Setting}
    \begin{itemize}
        \item Uniform Beta priors.
        \item Confidence in option $I = [1,3,4]$ at step $n$:
        \begin{align}
            \Theta_I &= \{\theta \in \Theta, \text{top-m}(\theta) = I\} \\
            \alpha_{I, n} &= \int_{\Theta_I} \pi_n(\theta) d\theta
        \end{align}
        \item Approximate via Monte Carlo method, $10^7$ samples $\mathcal{U}([0, 1]^{\#arms})$
        \item For fixed \#samples, compute confidence ($\alpha_{I, n}$).
        \item Average over 5-40 seeds.
        \item $\theta = [.1, .2, .3, .4, .5]$
        \item m = 3.
    \end{itemize} 
\end{frame}

\begin{frame}{Simulation: Observations}
    \begin{itemize}
        \item $\sim$ 'goes in right direction'
        \item Hypothesis: Edge of TTTS over uniform distribution should be greater when $\#arms$ and $m$ are 'further apart'.
    \end{itemize} 
\end{frame}

\end{document}